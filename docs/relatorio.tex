\documentclass[12pt]{article}
\usepackage[T1]{fontenc}
\usepackage{dot2texi}

\usepackage{tikz}
\usepackage{listings}

\usepackage{pgfplots}
\pgfplotsset{width=10cm,compat=1.9}

\usepackage{color}

\definecolor{mygreen}{rgb}{0,0.6,0}
\definecolor{mygray}{rgb}{0.5,0.5,0.5}
\definecolor{mymauve}{rgb}{0.58,0,0.82}

\lstset{ %
  backgroundcolor=\color{white},   % choose the background color; you must add \usepackage{color} or \usepackage{xcolor}
  basicstyle=\footnotesize,        % the size of the fonts that are used for the code
  breakatwhitespace=false,         % sets if automatic breaks should only happen at whitespace
  breaklines=true,                 % sets automatic line breaking
  captionpos=b,                    % sets the caption-position to bottom
  commentstyle=\color{mygreen},    % comment style
  deletekeywords={...},            % if you want to delete keywords from the given language
  escapeinside={\%*}{*)},          % if you want to add LaTeX within your code
  extendedchars=true,              % lets you use non-ASCII characters; for 8-bits encodings only, does not work with UTF-8
  frame=L,    	                   % adds a frame around the code
  keepspaces=true,                 % keeps spaces in text, useful for keeping indentation of code (possibly needs columns=flexible)
  keywordstyle=\color{blue},       % keyword style
  otherkeywords={*,...},           % if you want to add more keywords to the set
  numbers=left,                    % where to put the line-numbers; possible values are (none, left, right)
  numbersep=5pt,                   % how far the line-numbers are from the code
  numberstyle=\tiny\color{mygray}, % the style that is used for the line-numbers
  rulecolor=\color{black},         % if not set, the frame-color may be changed on line-breaks within not-black text (e.g. comments (green here))
  showspaces=false,                % show spaces everywhere adding particular underscores; it overrides 'showstringspaces'
  showstringspaces=false,          % underline spaces within strings only
  showtabs=false,                  % show tabs within strings adding particular underscores
  stepnumber=2,                    % the step between two line-numbers. If it's 1, each line will be numbered
  stringstyle=\color{mymauve},     % string literal style
  tabsize=2,	                   % sets default tabsize to 2 spaces
  title=\lstname                   % show the filename of files included with \lstinputlisting; also try caption instead of title
}

\usepackage{float}
\usepackage{graphicx}
\usepackage{caption}
\usepackage{subcaption}
\usepackage{wrapfig}
\usetikzlibrary{shapes,arrows}
\usepackage{bchart}

\makeatletter
%%%%%%%%%%%%%%%%%%%%%%%%%%%%%% Textclass specific LaTeX commands.
\newenvironment{lyxlist}[1]
{\begin{list}{}
{\settowidth{\labelwidth}{#1}
 \setlength{\leftmargin}{\labelwidth}
 \addtolength{\leftmargin}{\labelsep}
 \renewcommand{\makelabel}[1]{##1\hfil}}}
{\end{list}}

%%%%%%%%%%%%%%%%%%%%%%%%%%%%%% User specified LaTeX commands.
%\usepackage{sbc-template}

\usepackage[brazil]{babel}
\usepackage[utf8]{inputenc}

\usepackage[portuguese,ruled,vlined]{algorithm2e}
\usepackage{algorithmic}

\SetKwInput{Classe}{Classe}

\renewcommand{\abstractname}{Resumo}


\sloppy

\pagenumbering{arabic}

\title{Minimização de Valores de Arestas em um Grafo}

\author{Pedro Vanzella}

\makeatother

\usepackage{babel}
\usepackage{listings}
\lstset {
    mathescape,
    frame=none
}
\renewcommand{\lstlistingname}{Listagem}

\begin{document}

\maketitle

\begin{abstract}
  Uma solução para o problema de minimização do valor de imposto pago sobre
  transferências bancárias é apresentado. Utiliza-se um grafo para representar o
  conjunto de transferências, e minimiza-se o valor das arestas, através da
  busca por transitividades.
\end{abstract}

\section{Introdução}\label{sec:intro}

Uma recente mudança na regulamentação de impostos reativou uma antiga taxa sobre
operações financeiras. Esta taxa, chamada de CPMF, incide em $1\%$ sobre toda e
qualquer transação bancária.

Um banco teve a idéia de minimizar o valor total pago deste imposto através de
atalhos em transferências realizadas internamente.

Por exemplo, digamos que haja cinco correntistas, $1$, $2$, $3$, $4$
 e $5$, e haja as seguintes transferências entre eles:
\begin{list}{}{}
  \item $1$ transfere $\$500$ para $2$.
  \item $2$ transfere $\$230$ para $3$.
  \item $3$ transfere $\$120$ para $4$.
  \item $1$ transfere $\$120$ para $4$.
  \item $2$ transfere $\$200$ para $5$.
\end{list}

% TODO: Colocar estas duas listas em subfigures e referenciá-las

É possível fazer quatro transferências, respeitando os valores iniciais e finais
de saldo das contas destes cinco correntistas, mas minimizando o valor de cada
transferência, de modo a pagar menos imposto:

\begin{list}{}{}
  \item $1$ transfere $\$70$ para $2$
  \item $1$ transfere $\$110$ para $3$
  \item $1$ transfere $\$240$ para $4$
  \item $1$ transfere $\$200$ para $5$
\end{list}

Podemos ver que, em ambos os casos, o total enviado e o total recebido não foi
alterado~-~apenas os valores parciais mudaram e, com eles, o valor pago em impostos.

Do ponto de vista dos correntistas, nada mudou~-~\textit{e.g.} o extrato do
correntista $1$ ainda mostrará duas transferências, uma de $\$500$ para o
correntista $2$ e uma de $\$120$ para o correntista $4$~-~, mas internamente as
transferências realizadas foram bastante diferentes. Podemos ver uma
representação gráfica disto na Figura~\ref{fig:intro}.

\begin{figure}[htb!]
  \centering
  \begin{subfigure}[b]{0.5\textwidth}
    \begin{dot2tex}[neato,options=-tmath]
        \input{test_trivial.txt.input.dot}
    \end{dot2tex}
    \caption{Entrada}
    \label{fig:intro:entrada}
  \end{subfigure}%
  ~
  \begin{subfigure}[b]{0.5\textwidth}
    \begin{dot2tex}[neato,options=-tmath]
        \input{test_trivial.txt.reduced.dot}
    \end{dot2tex}
    \caption{Saída}
    \label{fig:intro:saída}
  \end{subfigure}
  \caption{Representação da entrada e da saída como grafos}
  \label{fig:intro}
\end{figure}

\pagebreak

\section{Entrada}\label{sec:entrada}

\begin{wrapfigure}{r}{0.3\textwidth}
  \vspace{-20pt}
  \begin{center}
    \begin{lstlisting}
    5 5
    1 2 500
    2 3 230
    3 4 120
    1 4 120
    2 5 200
    \end{lstlisting}
  \end{center}
  \caption{Arquivo de entrada}
  \vspace{-10pt}
\label{fig:entrada:file}
\end{wrapfigure}

O arquivo de entrada está no formato mostrado na Figura~\ref{fig:entrada:file}.
A primeira linha tem dois valores: a quantidade de correntistas e a quantidade
de transações descritas no arquivo. Como veremos na Seção~\ref{sec:estrutura},
estas informações não serão necessárias.

As linhas seguintes têm três valores cada: o correntista que originou a
transação, o correntista de destino da transação, e o valor da transação. Por
exemplo, na linha 2 da Figura~\ref{fig:entrada:file}, lemos ``\textit{Uma
transferência de 500 da conta do correntista 1 para a conta do correntista 2.}''.

Este arquivo de entrada representa o exemplo da Seção~\ref{sec:intro} e da
Figura~\ref{fig:intro:entrada}.

\section{Estrutura de Dados}\label{sec:estrutura}

A estrutura escolhida para representar o grafo foi a de Lista de Adjacências, como
mostrado na Figura~\ref{fig:estrutura:classes}.

Outras alternativas, como a representação em Matriz de Adjacências também são
possíveis, acarretando custos diferentes para acessos e remoções. No entanto, o
acesso aos adjacentes de um nodo por Lista de Adjacentes se dá em $O(adj)$,
enquanto por Matriz de Adjacências se dá em $O(n)$, onde $n$ é o número de nodos
do grafo. Como espera-se que haja menos arestas em um nodo do que nodos no
grafo, a performance da Lista de Adjacências tende a ser ligeiramente melhor
para esta aplicação.

No entanto, o fator principal para a escolha da representação por Lista de
Adjacências foi a facilidade de se trabalhar com ela.

\begin{figure}[H]
    \begin{lstlisting}
class Graph:
    public list<Node> nodes

    class Node:
        public int val
        public list<Edge> edges

    class Edge:
        public Node from
        public Node to
        public int val
    \end{lstlisting}
  \caption{Representação das classes do grafo}
\label{fig:estrutura:classes}
\end{figure}

Para ler o arquivo de entrada e criar os nodos e arestas, utilizamos o
Algoritmo~\ref{alg:estrutura:leitura}. Veja que ele está na classe
\textsf{\textbf{Mardita}}, que contém uma instância do grafo.

\begin{algorithm}[H]
 \caption{Criação de Nodos e Arestas}
 \label{alg:estrutura:leitura}
 \Classe{Mardita}
 \Entrada{Arquivo como o da Figura~\ref{fig:entrada:file}}
 \Saida{Instância da classe \textsf{\textbf{Graph}}}
 \ParaCada{linha l no arquivo, exceto a primeira}
    {
        partes $\gets$ l.separa(' ') \tcc*[r]{Separa a linha nos espaços}

        nodo\_a $\gets$ self.graph.add\_node(partes[0])

        nodo\_b $\gets$ self.graph.add\_node(partes[1])

        nodo\_a.add\_edge(nodo\_b, partes[2]) \tcc*[r]{Liga A com B, com valor partes[2]}
    }
\end{algorithm}

Onde \textsf{add\_node} está descrito no Algoritmo~\ref{alg:estrutura:add_node}
e \textsf{add\_edge} está descrito no Algoritmo~\ref{alg:estrutura:add_edge}.

\begin{algorithm}[H]
  \caption{Criação de Nodos}
  \label{alg:estrutura:add_node}
  \Classe{Graph}
  \Entrada{val: Inteiro, representando o nome do Nodo}
  \Saida{Instância da classe \textbf{\textsf{Node}}}
  \ParaTodo{n em self.nodes}
  {
    \Se{n.val $=$ val}
    {
      \Retorna{n} \tcc*[r]{Se o nodo já existe, retorna ele}
    }
    n $\gets$ Graph.Node(val) \tcc*[r]{Chama o construtor de Node}
    self.nodes.add(n) \tcc*[r]{Adiciona à lista de nodos}
    \Retorna{n}
  }
\end{algorithm}

O Algoritmo~\ref{alg:estrutura:add_node},
que pertence à classe \textsf{\textbf{Graph}}, primeiro verifica se já há um
nodo com este nome em sua coleção de nodos. Caso haja, retorna ele. Se não
houver, chama o construtor da classe \textsf{\textbf{Node}} para criar um novo
nodo, adiciona à sua coleção e então retorna o nodo criado.

A primeira vista, poderíamos ter utilizado um \textit{set} em vez de uma lista
para armazenar a coleção de nodos, dado que não queremos dois nodos iguais nela.
No entanto, a unicidade garantida seria do objeto nodo, quando queremos na
verdade a unicidade do nome do nodo.

\begin{algorithm}[H]
  \caption{Criação de Arestas}
  \label{alg:estrutura:add_edge}
  \Classe{Node}
  \Entrada{to: Nodo de origem; val: Inteiro, representando o valor da aresta}
  \Saida{Instancia da classe \textsf{\textbf{Edge}}}
  \ParaCada{e em self.edges}
  {
    \Se{e.to $=$ to}
    {
      \Retorna{e.update(val)} \tcc*[r]{Se a aresta já existe, aumenta seu valor}
    }
    e $\gets$ Graph.Edge(self, to, val) \tcc*[r]{Cria nova Edge}
    self.edges.add(e) \tcc*[r]{Adiciona à coleção de arestas}
    \Retorna{e}
  }
\end{algorithm}

O Algoritmo~\ref{alg:estrutura:add_edge} é
parecido com o Algoritmo~\ref{alg:estrutura:add_node}, pois ele verifica a unicidade da
aresta. A diferença é que arestas são consideradas iguais, neste problema caso
suas origens e destinos sejam iguais. Como estamos verificando todas as arestas
que partem de um nodo, basta comparar o destino.

O construtor da aresta recebe três parâmetros: \textit{de onde}, \textit{para
  onde} e o \textit{valor} da aresta.

Caso a aresta já exista, soma-se o valor das transações, de modo a criar apenas
uma aresta entre dois nodos. Um exemplo disto pode ser visto na
Figura~\ref{fig:double_edge}. O grafo da Figura~\ref{fig:double_edge:bad}
representa duas transações, com a mesma origem e o mesmo destino. Ao somar-se o
valor de suas arestas, temos o grafo da Figura~\ref{fig:double_edge:good}. Este
é o que é utilizado pelo algoritmo.

\begin{figure}[htb!]
  \centering
  \begin{subfigure}[b]{0.5\textwidth}
    \begin{dot2tex}[neato,options=-tmath]
        \input{double_edge.dot}
    \end{dot2tex}
    \caption{Entrada}
    \label{fig:double_edge:bad}
  \end{subfigure}%
  ~
  \begin{subfigure}[b]{0.5\textwidth}
    \begin{dot2tex}[neato,options=-tmath]
        \input{double_edge_fixed.dot}
    \end{dot2tex}
    \caption{Saída}
    \label{fig:double_edge:good}
  \end{subfigure}
  \caption{Consolidação de arestas com mesma origem e destino}
  \label{fig:double_edge}
\end{figure}


\section{Algoritmo}\label{sec:algoritmo}

Há duas coisas a serem feitas para resolver o problema: precisamos calcular
quanto imposto é pago (Seção~\ref{sec:algoritmo:total_tax}) e reduzir o número
de arestas no grafo, bem como seus valores (Seção~\ref{sec:algoritmo:reduce}).

\subsection{Cálculo de Total de Imposto Pago}\label{sec:algoritmo:total_tax}

Este algoritmo é executado duas vezes~-~uma antes de reduzir-se as arestas, e
uma após, de modo a sabermos qual foi a economia.

\begin{algorithm}[H]
  \caption{Cálculo de Imposto Pago}
  \label{alg:total_tax}
  \Classe{Mardita}
  \Entrada{Todas as arestas do grafo}
  \Saida{Total de imposto pago}
  total $\gets$ $0$

  \ParaTodo{e em self.graph.edges}
  {
    total $\gets$ total $+$ e.valor
  }
  \Retorna{total $\times$ $0.01$}
\end{algorithm}

No Algoritmo~\ref{alg:total_tax} vemos como o total de imposto é
calculado. Apenas acumula-se o valor de todas as arestas e multiplica-se por
$0.01$, que é o percentual do imposto.

Nota-se que está acessando-se a propriedade \textsf{edges} da classe
\textsf{\textbf{Graph}}, mas a mesma não parecia ter acesso às arestas, conforme
visto na Figura~\ref{fig:estrutura:classes}.

De fato, a lista de arestas está na classe \textsf{\textbf{Node}}. Para termos
acesso a elas, basta termos um método na classe \textsf{\textbf{Graph}} que
itera por todos os nodos, coletando todas as arestas. A unicidade das arestas é
garantida no momento de inserção, então pode-se fazer como é visto no Algoritmo~\ref{alg:all_edges}.

\begin{algorithm}[H]
  \caption{Coleção de todas as arestas}
  \label{alg:all_edges}
  \Classe{Graph}
  \Entrada{Uma instância da classe \textsf{textbf{Graph}}}
  \Saida{Uma lista de instâncias da classe \textsf{\textbf{Edge}}}
  inicializa edges como uma lista vazia

  \ParaTodo{n em self.nodes}
  {
    \ParaTodo{e em n.edges}
    {
      adiciona e em edges
    }
  }

  \Retorna{edges}
\end{algorithm}

\subsection{Redução das Arestas}\label{sec:algoritmo:reduce}

Este é o algoritmo principal, onde o problema é de fato solucionado. O
pseudocódigo pode ser visto no Algoritmo~\ref{alg:reduce_edges}.

\begin{algorithm}[H]
  \caption{Redução de Arestas}
  \label{alg:reduce_edges}
  \Classe{Mardita}
  \ParaTodo{u em self.graph.nodes}
  {
    vs $\gets$ adjacentes de u

    \Enqto{vs não estiver vazia}
    {
      v $\gets$ vs.pop() \tcc*[r]{Remove um elemento de vs}
      \ParaTodo{a adjacente a v}
      {
        \Se{valor de <v, a> $<$ valor de <u, v>}
        {
          tmp $\gets$ valor de <v,a>;
          remove <v, a>;
          valor de <u,v> $\gets$ diminui de tmp;
          cria aresta <u, a> com valor tmp;
        }
        \Senao{
          \Se{v está em vs}
          {
            remove v de vs \tcc*[r]{v pode ter sido conectado novamente graças a
            um ciclo}
          }
          tmp $\gets$ valor de <u,v>;
          remove <u,v>;
          valor de <v,a> $\gets$ diminui de tmp;
          cria aresta <u,a> com valor tmp;
          vs $\gets$ a \tcc*[r]{Nodo a é agora é adjacentes de u};
        }
      }
    }
  }
\end{algorithm}

A idéia do Algoritmo~\ref{alg:reduce_edges} é encontrar transitividade entre os
nodos~-~isto é, para um grafo $G = \{n, e\}$ com $n = \{A, B, C\}$ e $e = \{(A, B, x),
  (B, C, y)\}$, gerar uma nova aresta $(A, C, x - y)$, remover a aresta $(B, C,
x)$ e alterar o valor da aresta $(A, B)$ para $y - x$.

Há dois casos possíveis: o valor da primeira aresta pode ser maior que o da
segunda (Figura~\ref{fig:reduce:abc:entrada}) ou o valor da primeira aresta pode
ser menor ou igual ao da segunda (Figura~\ref{fig:reduce:uva:entrada}).

Na Figura~\ref{fig:reduce:abc} vemos o primeiro caso. Aqui pega-se o valor da
aresta $(2,3)$, diminui-se ela da aresta $(1,2)$, remove-se $(2,3)$ e
cria-se $(1,3)$, com o valor que se tinha em $(1,2)$. O resultado disto, aplicado à
Figura~\ref{fig:reduce:abc:entrada} pode ser visto na Figura~\ref{fig:reduce:abc:saida}.

Caso já exista uma aresta $(1,3)$, isto não é um problema, já que o
Algoritmo~\ref{alg:estrutura:add_edge} cuida para que, ao adicionar uma aresta
já existente, o valor da aresta seja apenas aumentado.

\begin{figure}[H]
  \centering
  \begin{subfigure}[b]{0.5\textwidth}
    \begin{dot2tex}[neato,options=-tmath]
        \input{test_abc.input.dot}
    \end{dot2tex}
    \caption{Entrada}
    \label{fig:reduce:abc:entrada}
  \end{subfigure}%
  ~
  \begin{subfigure}[b]{0.5\textwidth}
    \begin{dot2tex}[neato,options=-tmath]
        \input{test_abc.reduced.dot}
    \end{dot2tex}
    \caption{Saída}
    \label{fig:reduce:abc:saida}
  \end{subfigure}
  \caption{Redução aplicada a um grafo simples.}
  \label{fig:reduce:abc}
\end{figure}

Na Figura~\ref{fig:reduce:uva} vemos o segundo caso. A entrada
(Figura~\ref{fig:reduce:uva:entrada}) tem uma aresta $(1,2)$ com valor inferior
à aresta $(2,3)$. Neste caso, pegamos o valor de $(1,2)$ e removemos esta
aresta. Então, subtraímos este valor da aresta $(2,3)$ e criamos a aresta
$(1,3)$ com este mesmo valor. Isto gera o resultado visto na
Figura~\ref{fig:reduce:uva:saida}.

Novamente não precisamos nos preocupar ao adicionar uma aresta com o caso da
mesma já existir, já que o Algoritmo~\ref{alg:estrutura:add_edge} já cuida disto.

Há ainda um outro detalhe neste segundo caso: é possível que a aresta
intermediária (no caso da Figura~\ref{fig:reduce:uva}, a aresta 2), ter sido
adicionada novamente à lista de adjacentes da primeira aresta (pelo
\textsf{senão} do primeiro caso). Neste caso, é importante verificarmos a
existência dela e removê-la da lista, para evitarmos acessar um ponteiro
inválido ao tentar encontrar uma aresta entre \textsf{u} e \textsf{v} numa
iteração futura do laço \textsf{enquanto}.

\begin{figure}[H]
  \centering
  \begin{subfigure}[b]{0.5\textwidth}
    \begin{dot2tex}[neato,options=-tmath]
      \input{test_uva.input.dot}
    \end{dot2tex}
    \caption{Entrada}
    \label{fig:reduce:uva:entrada}
  \end{subfigure}%
  ~
  \begin{subfigure}[b]{0.5\textwidth}
    \begin{dot2tex}[neato,options=-tmath]
      \input{test_uva.reduced.dot}
    \end{dot2tex}
    \caption{Entrada}
    \label{fig:reduce:uva:saida}
  \end{subfigure}
  \caption{Redução aplicada a um grafo onde o valor de <v,a> é superior ao de
    <u,v>}
  \label{fig:reduce:uva}
\end{figure}

Pode-se facilmente validar o algoritmo com um teste de mesa na
Figura~\ref{fig:reduce:abc}, na Figura~\ref{fig:reduce:uva}, ou mesmo na
Figura~\ref{fig:intro}.

Olhando-se para a Figura~\ref{fig:reduce:abc}, vemos que os saldos dos
correntistas são os mesmos (\textit{i.e.}, o correntista $1$ teve uma redução de
$100$ em seu saldo, o correntista $2$ teve um aumento de $50$, bem como o
correntista $3$). A única coisa que mudou entre a
Figura~\ref{fig:reduce:abc:entrada} e a Figura~\ref{fig:reduce:abc:saida} foram
os valores e os destinos das transferências. Somando-se e multiplicando pelo
valor do imposto, vemos que na Figura~\ref{fig:reduce:abc:entrada} pagaríamos
$1.50$ de imposto (\textit{i.e.}, $1\%$ de $150$), enquanto na
Figura~\ref{fig:reduce:abc:saida} pagaríamos $1.00$ de impost (\textit{i.e.},
$1\%$ de $100$), o que representa uma economia de $0.50$.

\section{Resultados}\label{sec:resultados}

Podemos ver na Tabela~\ref{tab:resultados} a economia em cada caso de teste, bem
como uma comparação da quantidade de transações (arestas no grafo) com a
quantidade de iterações feitas pelo Algoritmo~\ref{alg:reduce_edges}. Na
Figura~\ref{fig:graph:arestas_iteracoes} podemos ver um gráfico plotando esta comparação.

\begin{table}[H]
  \centering
\begin{tabular}{||c|c||c|c|c||}
  \hline
Teste & Economia & Transações & Iterações \\ [0.5ex]
  \hline\hline
  1 & 5321.10 & 264 & 2533 \\
  \hline
  2 & 4077.60 & 235 & 2648 \\
  \hline
  3 & 2478.32 & 165 & 304 \\
  \hline
  4 & 2225.54 & 140 & 222 \\
  \hline
  5 & 1462.09  & 99 & 155 \\
  \hline
  6 & 2921.33 & 186 & 884 \\
  \hline
  7 & 1033.03  & 91 & 162 \\
  \hline
  8 & 4124.53 & 232 & 528 \\
  \hline
  9 & 2791.65 & 165 & 320 \\
  \hline
  10 & 2428.67 & 153 & 280 \\
\hline
\end{tabular}

\caption{Resultados dos testes da Turma 128}
\label{tab:resultados}
\end{table}

\begin{tikzpicture}
\begin{axis}[
    title={Número de Arestas vs Número de Iterações},
    xlabel={Arestas},
    ylabel={Iterações},
    xmin=0, xmax=300,
    ymin=0, ymax=3000,
    xtick={0,25,50,75,100,125,150,175,200,225,250,275,300},
    ytick={0,200,400,600,800,1000,1200,1400,1600,1800,2000,2200,2400,2600,2800,3000},
    legend pos=north west,
    ymajorgrids=true,
    grid style=dashed,
]

\addplot [
     domain=0:300,
     samples=100,
     color=red,
    ]{0.000120509*x^3 + 0.0460933*x^2 - 14.8593*x + 1046.9};
    \addlegendentry{$O(e^3)$}

\addplot[
    color=blue,
    mark=square,
    ]
    coordinates {
    (91,192)(99,155)(140,222)(153,280)(165,320)(165,304)(186,884)(232,528)(235,2648)(264,2533)
    };
    \addlegendentry{Teste}

\end{axis}
\end{tikzpicture}
\captionof{figure}{Arestas vs Iterações}
\label{fig:graph:arestas_iteracoes}

Avaliar a complexidade da solução é difícil, dado que os laços mais internos
dependem de valores que mudam com cada iteração e com cada entrada (\textit{i.e.
depende da quantidade de arestas que saem de cada nodo}). No entanto, uma função
cúbica, como $O(e^3)$ parece se ajustar moderadamente bem aos pontos, quando
ignoramos algumas constantes e variações.

Como todos os casos de teste tinham a mesma quantidade de nodos, não foi
possível avaliar a dependência da complexidade do problema na quantidade de nodos.

No entanto, ao analisarmos o Algoritmo~\ref{alg:reduce_edges}, vemos três laços
aninhados, o que é tipicamente uma marca de uma função cúbica. De fato, a
complexidade parece depender do número de nodos adjacentes a cada nodo, em algo
como $O(n) \times O(adj(n)) \times O(adj(adj(n)))$.

\section{Conclusão}\label{sec:conclusao}

É possível ver na Tabela~\ref{tab:resultados} que há uma economia significativa
em cada um dos casos de teste.

É possível que a escolha de Matriz de Adjacência seja melhor do que Lista de
Adjacência para a resolução deste problema, dado que a complexidade de buscar,
inserir ou remover uma aresta com Matriz de Adjacência é $O(n)$, onde $n$ é o
número de nodos do grafo, enquanto com Lista de Adjacência a complexidade destas
operações é $O(adj(u))$, onde $adj(u)$ é o número de nodos adjacentes a um nodo $u$.

No entanto, o uso de memória de Lista de Adjacência cresce linearmente com a
quantidade de nodos e arestas ($(O(n + e))$), enquanto o uso de memória da
Matriz de Adjacência cresce na ordem de $O(n^2)$.

No fim, esta escolha é irrelevante para o tamanho do problema, e a escolha de
utilizar-se Lista de Adjacências tornou o código mais limpo e fácil de se compreender.

\end{document}
